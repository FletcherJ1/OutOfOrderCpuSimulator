\documentclass[]{article}
\usepackage{bytefield}
\usepackage{longtable}

\setlength{\tabcolsep}{2pt}

%opening
\title{ISA Specification}
\author{James Fletcher}

\begin{document}

\maketitle

\tableofcontents

\begin{abstract}

\end{abstract}

\section{Summary}
A fictitious RISC Store-Load 32-bit CPU architecture specification for an out-of-order, speculative and superscalar CPU simulator. The CPU has 16 general-purpose 32-bit registers. The instruction set is based upon RISC-V and MIPS. Register 0 is wired to zero value and any writes to register 0 will be silently ignored.

\section{Instruction Formats}
There are 3 different formats of instructions, all 32 bits in size, as outlined below. All have a 7 bit opcode field placed at the MSB \footnote{MSB - Most Significant Byte}. A key is also provided to describe field names. The instruction format isn't optimal however it is easy to encode and understand which is the primary objective. \vspace{2ex}

\subsection{Field Key}
\begin{tabular}{|c|c|} 
	\hline
	Field Name & Description \\
	\hline
	Op & 6 Operation \\ 
	\hline
	Src1 & Source 1 Register \\
	\hline
	Src2 & Source 2 Register \\
	\hline
	Dst & Destination Register \\
	\hline
	Variable & Depends on operation \\
	\hline
	Constant & A constant value \\
	\hline
\end{tabular}

\subsection{Instruction Format 1 - A}
\begin{bytefield}[endianness=big,bitwidth=.8em]{32}
	\bitheader{0-31} \\
	\bitbox{7}{Op} & \bitboxes{1}{0000000000000} & \bitbox{4}{Dst} & \bitbox{4}{Src1} & \bitbox{4}{Src2}
\end{bytefield}
\subsection{Instruction Format 2 - B}
\begin{bytefield}[endianness=big,bitwidth=.8em]{32}
	\bitheader{0-31} \\
	\bitbox{7}{Op} & \bitbox{17}{Variable} & \bitbox{4}{Src1} & \bitbox{4}{Src2}
\end{bytefield}
\subsection{Instruction Format 3 - C}
\begin{bytefield}[endianness=big,bitwidth=.8em]{32}
\bitheader{0-31} \\
\bitbox{7}{Op} & \bitbox{21}{Constant} \bitbox{4}{Dst}
\end{bytefield}

\section{Instruction List}
The table below gives an overview of the instructions supported by the processor as well as an example of what the instruction would look like. Further on, each instruction's pseudo-code implementation is provided to help aid understanding. The cycle count for each instruction may not be 100\% accurate due to external events like branch misses, memory latency and other quirks of instructions, like multiplying by zero. Instruction names that end in a "u" are unsigned operations else they are assumed signed.

\begin{longtable}{|c|c|l|c|l|p{5cm}|}
	\hline
	Op & Instruction & C & I Form & Example & Description \\
	\hline
	\multicolumn{5}{|c|}{Load Instructions - "Var" is memory offset constant} \\
	\hline
	0h & lw Src1, Src2, Var & 1 & B & lw r1, r2, 0 & Load Word \\
	\hline
	2h & lb Src1, Src2, Var & 1 & B & lb r1, r2, -4 & Load Byte\\
	\hline
	3h & lbu Src1, Src2, Var & 1 & B & lbu r3, r2, 4 & Load Byte Unsigned \\
	\hline
	4h & lh Src1, Src2, Var & 1 & B & lh r1, r15, 2 & Load Half-Word \\
	\hline
	5h & lhu Src1, Src2, Var & 1 & B & lhu r5, r8, 0 & Load Half-Word Unsigned \\
	\hline
	\multicolumn{6}{|c|}{Store Instructions - "Var" is memory offset constant} \\
	\hline
	6h & sw Src1, Src2, Var & 1 & B & sw r2, r1, -4 & Store Word \\
	\hline
	7h & sb Src1, Src2, Var & 1 & B & sb r2, r1, 2 & Store Byte \\
	\hline
	8h & sh Src1, Src2, Var & 1 & B & sh r2, r1, 0 & Load Half-Word \\
	\hline
	\multicolumn{6}{|c|}{Load Immediate Instructions} \\
	\hline
	9h & ldi Dst, Const & 1 & C & ldi r1, 1024 & Load const to reg \\
	\hline
	Ah & ldhi Dst, Const & 1 & C & ldhi r3, 0 & Load const to upper 11-bits of reg \\
	\hline
	\multicolumn{6}{|c|}{Comparison Instructions - Var is a const value} \\
	\hline
	Bh & slt Dst, Src1, Src2 & 2 & A & slt r5, r8, r9 & Set Dst if Src1 $<$ Src2 \\
	\hline
	Ch & sltu Dst, Src1, Src2 & 2 & A & sltu r5, r8, r9 & Set Dst if Src1 $<$ Src2 \\
	\hline
	Dh & slti Src1, Src2, Var & 2 & B & slti r5, r8, -2 & Set Src1 if Src2 $<$ Var \\
	\hline
	Eh & sltiu Src1, Src2, Var & 2 & B & sltiu r5, r8, 5 & Set Src1 if Src2 $<$ Var \\
	\multicolumn{6}{|c|}{Control Flow Instructions} \\
	\hline
	Fh & beq Src1, Src2, Var & 3 & B & beq r5, r8, -2 & Branch if Src1 $=$ Src2 to PC += Var \\
	\hline
	10h & bne Src1, Src2, Var & 3 & B & bne r5, r8, -2 & Branch if Src1 $!=$ Src2 to PC += Var \\
	\hline
	11h & blt Src1, Src2, Var & 3 & B & blt r5, r8, -2 & Branch if Src1 $<$ Src2 to PC += Var \\
	\hline
	12h & bge Src1, Src2, Var & 3 & B & bge r5, r8, -2 & Branch if Src1 $>$ Src2 to PC += Var \\
	\hline
	13h & bltu Src1, Src2, Var & 3 & B & bltu r5, r8, -2 & Branch if Src1 $<$ Src2 to PC += Var \\
	\hline
	14h & bgeu Src1, Src2, Var & 3 & B & bgeu r5, r8, -2 & Branch if Src1 $>$ Src2 to PC += Var \\
	\hline
	15h & jal Dst, Var & 2 & C & jal r5, -2 & Dst = PC + 4, PC += Var \\
	\hline
	16h & jalr Src1, Src2, Var & 3 & B & jalr r5, r8, -2 & Src1 = PC + 4, PC = Src2 + Var \\
	\hline
	17h & j Const & 1 & C & j label & PC += Const \\
	\hline
	18h & jr Dst, Var & 2 & B & jr r2, 0 & PC = Dst + Const \\
	\hline
	\multicolumn{6}{|c|}{Arithmetic Instructions} \\
	\hline
	19h & add Dst, Src1, Src2 & 1 & A & add r1, r2, r3 & Dst = Src1+Src2 \\
	\hline
	1Ah & addi Dst, Src1, Const & 1 & B & addi r1, r2, 3 & Dst = Src1+Const \\
	\hline
	1Bh & sub Dst, Src1, Src2 & 1 & A & sub r1, r2, r3 & Dst=Src1-Src2 \\
	\hline
	1Ch & subi Dst, Src1, Const & 1 & B & subi r1, r2, 2 & Dst=Src1-Const \\
	\hline
	1Dh & mul Dst, Src1, Src2 & 7 & A & mul r1, r2, r3 & Dst = Src1*Src2 \\
	\hline
	1Eh & mulh Dst, Src1, Src2 & 7 & A & mulh r1, r2, r3 & Dst = (Src1*Src2)$>>$32 \\
	\hline
	1Fh & mulhsu Dst, Src1, Src2 & 7 & A & mulhsu r1, r2, r3 & Dst = (Src1*Src2)$>>$32 \\
	\hline
	20h & mulhu Dst, Src1, Src2 & 7 & A & mulhu r1, r2, r3 & Dst = (Src1*Src2)$>>$32 \\
	\hline
	21h & div Dst, Src1, Src2 & 12 & A & div r1, r2, r3 & Dst = Src1/Src2 \\
	\hline
	22h & divu Dst, Src1, Src2 & 12 & A & divu r1, r2, r3 & Dst = Src1/Src2 \\
	\hline
	23h & rem Dst, Src1, Src2 & 12 & A & rem r1, r2, r3 & Dst = Src1\%Src2 \\
	\hline
	24h & remu Dst, Src1, Src2 & 12 & A & remu r1, r2, r3 & Dst = Src1\%Src2 \\
	\hline
	\multicolumn{6}{|c|}{Logic Instructions} \\
	\hline
	25h & and Dst, Src1, Src2 & 1 & A & and r1, r2, r3 & Dst = Src1\&Src2 \\
	\hline
	26h & or Dst, Src1, Src2 & 1 & A & or r1, r2, r3 & Dst = Src1$|$Src2 \\
	\hline
	27h & xor Dst, Src1, Src2 & 1 & A & xor r1, r2, r3 & Dst = Src1\string^Src2 \\
	\hline
	28h & andi Dst, Src1, Const & 1 & B & andi r1, r2, 1 & Dst = Src1\&Const \\
	\hline
	29h & ori Dst, Src1, Const & 1 & B & ori r1, r2, 2 & Dst = Src1$|$Const \\
	\hline
	2Ah & xori Dst, Src1, Const & 1 & B & xori r1, r2, r3 & Dst = Src1\string^Const \\
	\hline
	2Bh & srl Dst, Src1, Src2 & 1 & A & srl r1, r2, r3 & Dst = Src1$>>$Src2 \\
	\hline
	2Ch & srli Dst, Src1, Const & 1 & B & srli r1, r2, 31 & Dst = Src1$>>$Const \\
	\hline
	2Dh & sll Dst, Src1, Src2 & 1 & A & sll r1, r2, r3 & Dst = Src1$<<$Src2 \\
	\hline
	2Eh & slli Dst, Src1, Const & 1 & B & slli r1, r2, 2 & Dst = Src1$<<$Const \\
	\hline
	2Fh & srai Dst, Src1, Const & 1 & B & srai r1, r2, 1 & Dst = Src1$<<<$Const \\
	\hline
	30h & sra Dst, Src1, Src2 & 1 & A & sra r1, r2, r3 & Dst = Src1$<<<$Src2 \\
	\hline
	\multicolumn{6}{|c|}{Experimental Instructions} \\
	\hline
	XXh & pushb Dst, Src1 & 2 & A & pushb r1, r2 & mem[Dst] = Src1[0:7], Dst+=1 \\
	\hline
	XXh & pushh Dst, Src1 & 2 & A & pushh r1, r2 & mem[Dst] = Src1[0:15], Dst+=2 \\
	\hline
	XXh & push Dst, Src1 & 2 & A & push r1, r2 & mem[Dst] = Src1, Dst+=4 \\
	\hline
	XXh & popb Dst, Src1 & 2 & A & popb r1, r2 & Dst-=1, Src1[0:7]=mem[Dst] \\
	\hline
	XXh & poph Dst, Src1 & 2 & A & poph r1, r2 & Dst-=2, Src1[0:15]=mem[Dst] \\
	\hline
	XXh & pop Dst, Src1 & 2 & A & pop r1, r2 & Dst-=4, Src1=mem[Dst] \\
	\hline
\end{longtable}

\section{Memory}
This defines how memory is interacted with and is currently not decided on.
Total Store Ordering? Weak Memory Model? Strong Memory Model?

\end{document}
